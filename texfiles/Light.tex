\chapter{Light} 
\label{ch:Light}

\section{Quantum Treatment of Light}

Wave particle duality.

There are two approaches to modelling light: classical and quantum. 

When quantizing the electromagnetic field it is convenient to work in terms of creation and annihilation operators, $\cre{a}_\lambda(\vec{k})$ and $\an{a}_\lambda(\vec{k})$, for field modes
\begin{align}
  u(\vec{k}; \vec{r}, t) \propto e^{i\vec{k} \cdot \vec{r} - i\omega_\vec{k} t}
\end{align}
with polarization $\lambda$. The mode $u(\vec{k}; \vec{r}, t)$ describes a plane wave with definite momentum $\vec{k}$, which is unphysical due to its infinite extend in space. Nevertheless these modes are very useful when describing light quantum mechanically.

In terms of these operations, the Hamiltonian for the free field is given by
\begin{align}
  H &= \sum_\lambda \int d\vec{k} \frac{\hbar\omega_k}{2} \left[ \cre{a}_\lambda(\vec{k})\an{a}_\lambda(\vec{k}) + \an{a}_\lambda(\vec{k})\cre{a}_\lambda(\vec{k}) \right] \\
  &= \sum_\lambda \int d\vec{k} H_\lambda(\vec{k}).
\end{align}

The creation and annihilation operators are so-called as they can be thought to as adding or removing an excitation in a given mode. Commutation relation $[\an{a}_\lambda(\vec{k}), \cre{a}_{\lambda '}(\vec{k}')]  = \delta_{\lambda\lambda '}\delta^3(\vec{k} - \vec{k}') $.

We have that
\begin{align}
  \left[H, \an{a}_\lambda(\vec{k})\right] = -\hbar \omega_\vec{k} \an{a}_\lambda(\vec{k})
  \quad \text{and} \quad 
  \left[H, \cre{a}_\lambda(\vec{k})\right] = \hbar \omega_\vec{k} \cre{a}_\lambda(\vec{k})
\end{align}

Suppose we have a state $\ket{\psi}$ which is an eigenstate of the Hamiltonian with energy $E$. Consider the state $\cre{a}\ket{\psi}$:
\begin{align}
  H\cre{a}_\lambda(\vec{k})\ket{\psi} &= \left(\hbar \omega_\vec{k} \cre{a}_\lambda(\vec{k}) + \cre{a}_\lambda(\vec{k})H\right) \ket{\psi} \\
  &= \left(\hbar \omega_\vec{k} + E\right) \cre{a}_\lambda(\vec{k})\ket{\psi}
\end{align}
so $\cre{a}_\lambda(\vec{k})\ket{\psi}$ is also an eigenstate of the Hamiltonian, this time with energy $E + \hbar\omega_\vec{k}$. The operator $\cre{a}_\lambda(\vec{k})$ has created an excitation with energy $\hbar \omega_\vec{k}$.

If we start with a vacuum state of the electomagnetic field $\ket{0}$ such that $H\ket{0} = 0$ [TODO?? is this right] we can use the operators $\cre{a}$ to define set of states:
\begin{align}
  \ket{n} = (\cre{a})^n \ket{0}
\end{align}
These are called the \textit{Fock states} and corespond to states with a definite number of excitations.These are called the \textit{Fock states} and corespond to states with a definite number of excitations.  


Single photon states

\subsection{Hong-Ou-Mandel Dip}

The Hong-Ou-Mandel experiment concerns the behaviour of a pair of photons iteracting with a \textit{beam splitter}. A beam splitter is an optical device that reflects some of the incident light while allowing the rest to pass through. Given two optical modes $a_1$ and $a_2$ the beam splitter performs the transformation
\begin{align}
  \an{a}_{1} \rightarrow \an{a}_{1} + \an{a}_{2} \\
  \an{a}_{2} \rightarrow \an{a}_{1} - \an{a}_{2}
\end{align}
If both modes are incident on the beam splitter simultateously,
\begin{align}
  \an{a}_{1}\an{a}_{2} \rightarrow \left(\an{a}_{1} + \an{a}_{2}\right) \left(\an{a}_{1} - \an{a}_{2}\right) = \an{a}_{1}\an{a}_{1} - \an{a}_{2}\an{a}_{2}
\end{align}
The output states contain only those cases where both exitations are in the same mode.

So if we have two single mode exitations of a given frequency incident on different arms of a beam splitter they exihit a `bunching' behaviour: they will leave on the same arm. As already discussed, single mode exitations are not physical. Does a similar result apply to a pair of single photons arriving at a beam splitter?

A single photon can be written in terms of its annihilation operator
\begin{align}
  \an{b} = \int f(k) \an{a} dk
\end{align}

\begin{align}
  \an{b}_1 \an{b}_2 & = \int \int f_1(k)f_2(k') \an{a}_1(k) \an{a}_2(k') dkdk' \\
                    & \rightarrow
\end{align}

\subsection{Input-Output formalism}



\section{Light-Matter interactions}

\subsection{Atom in a cavity}

Atom couples to the electric field through the dipole coupling $H_d = -\vec{d}\cdot\vec{E}(t)$ where $\vec{d} = -e \vec{r}$ for the charge $e$ and the position operator $\vec{r}$.

We look at the case of an atom in a cavity. In a cavity only a discrete set of modes can exist, which simplifies the calculation.

We also assume that the lowset mode frequency is nearly resonant with the atomic transition, which will allow us to ignore higher modes later in the calculation.

Look at a two level system with Hamiltonian
\begin{align}
  H = -\frac{\omega_0}{2}\ket{g}\bra{g} +\frac{\omega_0}{2}\ket{e}\bra{e}
\end{align}
We call $\ket{g}$ the ground state and $\ket{e}$ the excited state. 


Classically the electric field $\vec{E}$ is given
\begin{align}
  \vec{E} = E_0 (\vec{\epsilon}\exp(i[\omega t - \vec{k}\cdot\vec{n}r]) + \vec{\epsilon}^*\exp(-i[\omega t - \vec{k}\cdot\vec{n}r])
\end{align}

We assume that we may treat the atom's center of mass classically and that the atom is positioned at $r=0$.

We need to write the dipole Hamiltonian, $H_d$, in terms of our system basis $\{\ket{g}, \ket{e}\}$
As the position operator $\vec{r}$ has odd parity, [??], the diagonal elements $\bra{g}H_d\ket{g}$ and $\bra{e}H_d\ket{e}$ must both be equal to $0$. As the Hamiltonian is Hermitian there is only one element left to specify:
\begin{align}
  \bra{e}H_d\ket{g} = \bra{g}H_d\ket{e}^* = E_0e\vec{r}_{eg}\cdot(\vec{\epsilon}\exp(i\omega t) + \vec{\epsilon}^*\exp(-i\omega t))
\end{align}
where $\vec{r}_{eg} = \bra{e}\vec{r}\ket{g}$

Over all this leaves us with the Hamiltonian
\begin{align}
  H =
  \begin{bmatrix}
    -\omega_0/2 & eE_0\vec{r}_{eg}^*\cdot(\vec{\epsilon}^*\exp(-i\omega t) + \vec{\epsilon}\exp(i\omega t)) \\
    eE_0\vec{r}_{eg}\cdot(\vec{\epsilon}\exp(i\omega t) + \vec{\epsilon}^*\exp(-i\omega t)) & +\omega_0/2
  \end{bmatrix}
\end{align}

Still time dependent. Make a unitary transformation to rotate with the EM field:
\begin{align}
  U(t) =
  \begin{bmatrix}
    e^{i\omega t/2} & 0 \\
    0 & e^{-i\omega t/2}
  \end{bmatrix}
\end{align}

This gives us
\begin{align}
  H =
  \begin{bmatrix}
    -\omega_0/2 & eE_0\vec{r}_{eg}^*\cdot(\vec{\epsilon}^* + \vec{\epsilon}\exp(2i\omega t)) \\
    eE_0\vec{r}_{eg}\cdot(\vec{\epsilon} + \vec{\epsilon}^*\exp(-2i\omega t)) & +\omega_0/2
  \end{bmatrix}
\end{align}

We now make the rotating wave approximation, neglecting the fast rotating terms $e^{2i\omega t}$ and $e^{-2i\omega t}$ on the basis that they will be far off-resonance and their contribution will average to zero over time. 

We are left with
\begin{align}
  H =
  \begin{bmatrix}
    -\omega_0/2 & eE_0\vec{r}_{eg}^*\cdot\vec{\epsilon}^* \\
    eE_0\vec{r}_{eg}\cdot\vec{\epsilon} & +\omega_0/2
  \end{bmatrix}
\end{align}
which is of the form of the Rabi Hamiltonian (\ref{rabi_hamiltonian}). An atom in a cavity interacting with an EM field will oscillate.

If we want to look at the emission and absorbtion of single photons we must instead use the fully quantised field. The electric field is then given 
\begin{align}
  \vec{E} = \sum_{\lambda = 1}^2 \sum_\vec{k} E_0 (\vec{\epsilon}_\lambda(\vec{k})\an{a}_\lambda(\vec{k})+ \vec{\epsilon}^*_\lambda(\vec{k})\cre{a}_\lambda(\vec{k})
\end{align}



\begin{align}
  H = \frac{\hbar \omega_0}{2} \sz + \hbar \omega \cre{b}\an{b} + g\an{b} \sp + g^{*}\cre{b} \sm
\end{align}

\subsection{Raman procedure}


\begin{align}
  H=\hbar
  \begin{bmatrix}
    0 & 0 & \Omega_1 \cos\omega_1 t \\
    0 & \delta & \Omega_2 \cos\omega_2 t \\
    \Omega_1 \cos\omega_1 t & \Omega_2 \cos\omega_2 t & \omega
  \end{bmatrix}
\end{align}

\begin{align}
  U(t) = 
  \begin{bmatrix}
    1 & 0 & 0 \\
    0 & e^{i(\omega_1 - \omega_2)t} & 0 \\
    0 & 0 & e^{\omega_1 t}
  \end{bmatrix}
\end{align}

\begin{align}
  \tilde{H}=\frac{\hbar}{2}
  \begin{bmatrix}
    0 & 0 & \Omega_1(1+e^{2i\omega_1 t}) \\
    0 & 2(\delta + \omega_2 - \omega_1) & \Omega_2 (1+e^{2i\omega_2 t}) \\
    \Omega_1(1+e^{2i\omega_1 t}) & \Omega_2 (1+e^{2i\omega_2 t}) & 2(\omega - \omega_1)
  \end{bmatrix}
\end{align}

After we make the rotating wave approximation
\begin{align}
  \tilde{H}=\frac{\hbar}{2}
  \begin{bmatrix}
    0 & 0 & \Omega_1 \\
    0 & 2(\nu_1 - \nu_2) & \Omega_2  \\
    \Omega_1 & \Omega_2  & 2\nu_1
  \end{bmatrix}
\end{align}

Look at the limit $\nu \gg \{\Omega_1, \Omega_2\}$.  Degenerate perturbation theory
\begin{align}
  H = -\frac{1}{4\nu}
  \begin{bmatrix}
    \Omega_1^2 & \Omega_1 \Omega_2 \\
    \Omega_1\Omega_2 & \Omega_2^2
  \end{bmatrix}
\end{align}
which is the basically the Rabi Hamiltonian as in Eq. (\ref{rabi_hamiltonian}).

Oscillation with angular frequency $\Delta = (\Omega_1^2 + \Omega_2^2)/4\nu$ induced via the third state.



\section{Phonon-matter interactions}

When dealing with solid state systems an important source of decoherence comes from interaction with lattice vibrations. Following a similar pattern to the quantisation of the electromagnetic field, we can quantise the vibrational modes of a lattice, in terms of \textit{phonons}, quanta of vibrational energy.

Phonons couple to any states of our system that lead to a distortion in the atomic lattice. We restrict ourselves to looking at relatively low-energy states, coupling only to long-wavelength, accoustic phonons. There are two dominant coupling mechanisms in this case: \textit{deformation potential} coupling, $D$, and \textit{piezoelectric coupling}, $P$, \cite{mahan} leading to the interaction Hamiltonian
\begin{align}
  H = \sum_\vec{q} \sqrt{\frac{\hbar}{2\mu V\omega_\vec{q}}} \left(D \vert \vec{q}\vert + iP \right)\hat{g}(\vec{q})\left(\an{a}_\vec{q} + \cre{a}_{-\vec{q}} \right).
\end{align}
[Find mahan and add something interesting]

We also restrict ourselves to looking at coupling to a phonon bath in a thermal state
\begin{align}
  \rho_E = \frac{1}{Z_\beta} e^{-\beta H_E}
\end{align}
where $\beta = 1/(kT)$ and
\begin{align}
  H_E = \sum_\alpha \omega_\alpha \cre{b}_\alpha \an{b}_\alpha
\end{align}

In this state we can calculate the correlation functions explicitly
\begin{align}
  \langle \an{b}_i \an{b}_j \rangle &= 0 \\
  \langle \cre{b}_i \cre{b}_j \rangle &= 0 \\
  \langle \an{b}_i \cre{b}_j \rangle &= \delta_{ij} \left( N(\omega_i) +1 \right) \\
  \langle \cre{b}_i \an{b}_j \rangle &= \delta_{ij} N(\omega_i)
\end{align}
where
\begin{align}
  N(\omega) \frac{1}{e^{\beta \omega} - 1}
\end{align}
is the occupation of state $i$.

When we take the sum over $\vec{q}$ we end up with 
\begin{align}
  J(\omega) = 2\pi \sum_\vec{q} |g_\vec{q}|^2 \delta(\omega-\omega_\vec{q})
\end{align}
so that we get
\begin{align}
  \gamma_\alpha(\omega) = \left\{
    \begin{array}{l l}
    J(\omega)\left(N(\omega) + 1\right) & \qquad \text{if $\omega > 0$}\\
    J(\omega)N(\omega) & \qquad \text{if $\omega \leq 0$}
  \end{array} \right.
\end{align}


