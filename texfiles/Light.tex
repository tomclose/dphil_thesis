\chapter{Light} 
\label{ch:Light}

\section{Quantum Treatment of Light}

The quest to define the nature of light occupies an important place in the history of physics. In some ways light appeared to behave like a particle and in other ways it appeared to behave like a wave. Newton favoured the particular description, which formed the basis for his work on lenses. This view was thought to have been largely discredited when Maxwell introduced his unifying equations for the electic field, which for $\vec{E}$ and magnetic field $\vec{B}$ in a vacuum reduce to:
\begin{align}
  \nabla \times \vec{B} &= \frac{1}{c^2} \frac{\partial \vec{E}}{\partial t} \label{maxwell_be} \\
  \nabla \times \vec{E} &= - \frac{\partial \vec{B}}{\partial t}\\
  \nabla \cdot \vec{B} &= 0 \\
  \nabla \cdot \vec{E} &= 0,
\end{align}
where $c$ is the speed of light. These equations leads to the wave equation for $\vec{E}(\vec{r}, t)$:
\begin{align}\label{wave_eq}
  \nabla^2\vec{E} - \frac{1}{c^2} \frac{\partial^2 \vec{E}}{\partial t^2} = 0.
\end{align}
and the wave/particle question was thought to be settled.

What changed their minds?


\subsection{Quantising Light: A toy model}

The normal approach to quantising the electromagnetic field involves working with the electromagnetic vector potential and is fairly technical due to the relativistic nature of light and continuous spectrum of modes. We aim to avoid these technical complexities while still providing an overview of the key stages by looking at a toy model of light: an electromagnetic field in a one dimensional cavity. We later state the results for the quantisation of the full field.

We look at a waves propagating in a single direction (which we take to be parallel to the z-axis), that are polarised in the $xz$ plane. To simplify the situation further we take $c = 1$ and the length of the cavity to be $1$. Solving the wave equation (\ref{wave_eq}) gives the only non-zero component of $\vec{E}$
\begin{align}\label{toy_classical_e}
  E_x &= \sum_j \sqrt{\frac{2\omega_j^2}{\epsilon_0 L}} q_j(t) \sin(k_j z) \\
  B_y &= \sum_j \frac{1}{k_j \mu_0} \sqrt{\frac{2\omega_j^2}{\epsilon_0 L}} q_j(t) \cos(k_j z),
\end{align}
where $\omega_j = \pi j$ for $j \in \{1, 2, \dots\}$. We know from the wave equation that
\begin{align}
  \ddot{q}_j = -\omega_j^2 q_j
\end{align}
which has solutions
\begin{align}
  q_j = Ae^{-i\omega_j t} + Be^{i\omega_j t} = ae^{-i\omega_j t} + a^*e^{i\omega_j t},
\end{align}
where in the last equality we have enforced that $q_j(t)$ must be real.

We can write the Hamiltonian 
\begin{align}
  H = \frac{\epsilon_0}{2} \int_0^L \vert \vec{E}\vert^2 + c^2\vert\vec{B}\vert^2 dz 
  = \frac{1}{2} \sum_j \left(\omega_j q_j^2 + p_j^2\right)
\end{align}
where $p_i = \dot{q_i}$ is the conjugate momentum for $q$. To quantize we promote $q_i$ and $p_i$ to operators, imposing the commutation relation
\begin{align}
  [q_i, p_j] = i\delta_{ij} \\
  [q_i, q_j] = [p_i, p_j] = 0.
\end{align}

Rewriting the amplitude relation gives
\begin{align}
  \hat{q}_i &= \sqrt{\frac{\hbar}{2\omega_i}}\left( \an{a}e^{-i\omega_i t} + \cre{a}e^{i\omega_i t}\right) \\
  \hat{p}_i &= -i \sqrt{\frac{\hbar\omega_i}{2}}\left( \an{a}e^{-i\omega_i t} + \cre{a}e^{i\omega_i t}\right).
\end{align}
The commutation relations for $\hat{x}$ and $\hat{p}$ imply the following relations for $\cre{a}$ and $\an{a}$:
\begin{align}
  [\cre{a}_i, \an{a}_j] = \delta_{ij} \\
  [\cre{a}_i, \cre{a}_j] = 0 \\
  [\an{a}_i, \an{a}_j] = 0
\end{align}

We can write the Hamiltonian in terms of the amplitude operators:
\begin{align}
  H = \sum_k \hbar \left( \omega_k \cre{a}\an{a} +\frac{1}{2} \right)
\end{align}
We have that
\begin{align}
  \left[H, \an{a}_k\right] = -\hbar \omega_k \an{a}_k
  \quad \text{and} \quad 
  \left[H, \cre{a}_k\right] = \hbar \omega_k \cre{a}_k
\end{align}

Suppose we have a state $\ket{n}$ which is an eigenstate of the Hamiltonian with energy $E_n$. By using the commutation relations for $H$ and $\an{a}$ we see that
\begin{align}
  H\an{a}_k\ket{n} = \left(E_n - \hbar \omega_k\right) \an{a}_k\ket{n}
\end{align}
so $\an{a}_k\ket{\psi}$ is also an eigenstate of the Hamiltonian, this time with energy $E - \hbar\omega_k$. We let 
\begin{align}
  \ket{n-1} = k_{n-1}\an{a} \ket{n}
\end{align}
where $k_{n-1}$ is a normalisation constant. By repeatedly applying $a$ we can create a ladder of states at an energy spacing of $\hbar\omega_k$. In order for the system to have a lowest energy level $E_0$ this process must terminate somehow. If $\ket{0}$ is an eigenstate of $H$ with energy $E_0$ we must have
\begin{align}
  a\ket{0} = 0
\end{align}
or otherwise $a\ket{0}$ would be an eigenstate with energy lower than $E_0$. We can use $\ket{0}$, which we call the \textit{vacuum state}, to construct the states $\ket{n}$ explictly:
\begin{align}
  \ket{n} = \frac{(\cre{a})^n \ket{0}}{\sqrt{n!}}. 
\end{align}
We then have that $E_n = (n+1/2)\hbar\omega$. Note that $E_0 = 1/2\hbar \omega$, so that the ground energy state is not zero. We ascribe this energy to \textit{vacuum fluctuations}.

The states $\ket{n}$ are commonly known as \textit{Fock states}. If we introduce the \textit{number operator},
\begin{align}
  N_k = \cre{a}_k\an{a}_k,
\end{align}
we have that
\begin{align}
  N_k\ket{n}_k = n.
\end{align}
We interpret the number operator as counting the number of excitations in the mode $k$. The operators $\an{a}_k$ are known as \textit{annihilation operators} and the operators $\cre{a}_k$ as \textit{creation operators}, as they can be thought of as respectively removing and adding excitations to the system. 

We can return to the classical expression for the electic field (\ref{toy_classical_e}) and promote it to a quantum field by writing in terms of creation and annihilation operators:
\begin{align}
  \hat{E}_x = \sum_j \left(\an{a}_k e^{-i\omega_k t} + \cre{a}_k e^{i\omega_k t}\right) \sin(k_j s).
\end{align}
Note that $\bra{n}\hat{E}_x\ket{n} = 0$ for all $n$, so that for a state with a definite number of excitations the expected value of the electric field at any point is zero. If we instead introduce states
\begin{align}
  \ket{\alpha} = \sum_n \frac{\alpha^n \ket{n}}{\sqrt{n!}}
\end{align}
for complex number $\alpha = p + iq$, we find that
\begin{align}
  \bra{\alpha}\hat{E}_x\ket{\alpha} = (p\cos\omega t + q\sin\omega t)\sin(k_j z).
\end{align}
The states $\ket{\alpha}$ are known as \textit{coherent states} and are the closest quantum analogue to the classical states of radiation.

\subsection{Field in free space}

We now present the corresponding results for the full electromagnetic field in free space.

Classically the state of the electromagnetic field can be decomposed into plane waves [TODO: in Kok + Lovett 1.12 do they need the sum over $\lambda$?]
\begin{align}
  \vec{E}(\vec{r}, t) = \sum_\lambda \int d\vec{k} a_\lambda(\vec{k})\vec{\epsilon}_\lambda u(\vec{k}; \vec{r}, t) + \text{c.c.},
\end{align}
where the modes $u(\vec{k}; \vec{r}, t)$ describe plane waves with definite momentum $\vec{k}$,
\begin{align}
  u(\vec{k}; \vec{r}, t) \propto e^{i\vec{k} \cdot \vec{r} - i\omega_\vec{k} t},
\end{align}
and $\vec{\epsilon}_\lambda$ is the polarization vector, which is perpendicular to $\vec{k}$.

After quantisation we obtain the expression
\begin{align}
  \hat{\vec{E}}(\vec{r}, t) = \sum_\lambda \int d\vec{k} \an{a}_\lambda(\vec{k})\vec{\epsilon}_\lambda u(\vec{k}; \vec{r}, t) + \text{H.c.},
\end{align}
where $\an{a}_\lambda(\vec{k})$ and $\cre{a}_\lambda(\vec{k})$ are the annihilation and creation operators for excitations of the mode $u(\vec{k}; \vec{r},t)$, which satisfy
\begin{align}
  [\an{a}_\lambda(\vec{k}), \cre{a}_{\lambda '}(\vec{k}')]  &= \delta_{\lambda\lambda '}\delta^3(\vec{k} - \vec{k}') \\
  [\an{a}_\lambda(\vec{k}), \an{a}_{\lambda '}(\vec{k}')]  &= 0 \\
  [\cre{a}_\lambda(\vec{k}), \cre{a}_{\lambda '}(\vec{k}')] &= 0.
\end{align}
The modes $u(\vec{k}; \vec{r}, t)$ are not physical, as the plane wave solutions cannot be normalised, but nevertheless the operators  $\an{a}_\lambda(\vec{k})$ and $\cre{a}_\lambda(\vec{k})$ are very convenient when formulating the theory.

In terms of these operators, the Hamiltonian for the free field is given by
\begin{align}
  H &= \sum_\lambda \int d\vec{k} \frac{\hbar\omega_k}{2} \left[ \cre{a}_\lambda(\vec{k})\an{a}_\lambda(\vec{k}) + \an{a}_\lambda(\vec{k})\cre{a}_\lambda(\vec{k}) \right]
\end{align}


\subsection{Hong-Ou-Mandel Dip}

The Hong-Ou-Mandel experiment concerns the behaviour of a pair of photons iteracting with a \textit{beam splitter}. A beam splitter is an optical device that reflects some of the incident light while allowing the rest to pass through. Given two optical modes $a_1$ and $a_2$ the beam splitter performs the transformation $B$ where
\begin{align}
  B (\cre{a}_{1}) = \frac{1}{\sqrt{2}} \left(\cre{a}_{1} + \cre{a}_{2}\right) \\
  B (\cre{a}_{2}) = \frac{1}{\sqrt{2}} \left(\cre{a}_{1} - \cre{a}_{2}\right).
\end{align}
If both modes are incident on the beam splitter simultateously,
\begin{align}
  B\left(\cre{a}_{1}\cre{a}_{2}\right) = \left(\cre{a}_{1} + \cre{a}_{2}\right) \left(\cre{a}_{1} - \cre{a}_{2}\right) = \cre{a}_{1}\cre{a}_{1} - \cre{a}_{2}\cre{a}_{2}
\end{align}
The output states contain only those cases where both exitations are in the same mode.

So if we have two single mode exitations of a given frequency incident on different arms of a beam splitter they exihit a `bunching' behaviour: they will leave on the same arm. As already discussed, single mode exitations are not physical. Does a similar result apply to a pair of single photons arriving at a beam splitter?

A single photon can be written in terms of its annihilation operator
\begin{align}
  \cre{b}\ket{0} = \int f(k) \cre{a}(k) dk\ket{0}
\end{align}
where we require that $\int f(k)^* f(k) dk = 1$.

\begin{align}
  \ket{\psi_\text{out}} &= B(\cre{b}_1 \cre{b}_2)\ket{0}\\
  & = \int \int f_1(k)f_2(k') B(\cre{a}_1(k) \cre{a}_2(k')) dkdk' \\
  & = \int \int f_1(k)f_2(k') \left(\cre{a}_{1}(k) + \cre{a}_{2}(k)\right) \left(\cre{a}_{1}(k') - \cre{a}_{2}(k')\right)  dkdk' \\
\end{align}

The probability of detecting the two photons in different arms of the detector is given by
\begin{align}
  C = \bra{\psi_\text{out}} \int\int \cre{a}_1(k)\an{a}_1(k) \cre{a}_2(k')\an{a}_2(k') dk dk'\ket{\psi_\text{out}}.
\end{align}
Calculating this for our output state gives
\begin{align}
  C = \frac{1}{2} - \frac{1}{2}\int\int f_1(k)f_1^*(k')f_2(k')f_2^*(k) dk dk'.
\end{align}
If $f_1 = f_2$ the $C=0$: perfectly identical photon wavepackets demonstrate perfect bunching at a beam splitter. This allows us to use $C$ as a measure of how identical two photons are.

\section{Light-Matter interactions}

\subsection{Atom in a cavity}

Atom couples to the electric field through the dipole coupling $H_d = -\vec{d}\cdot\vec{E}(t)$ where $\vec{d} = -e \vec{r}$ for the charge $e$ and the position operator $\vec{r}$.

We look at the case of an atom in a cavity. In a cavity only a discrete set of modes can exist, which simplifies the calculation.

We also assume that the lowset mode frequency is nearly resonant with the atomic transition, which will allow us to ignore higher modes later in the calculation.

Look at a two level system with Hamiltonian
\begin{align}
  H = -\frac{\omega_0}{2}\ket{g}\bra{g} +\frac{\omega_0}{2}\ket{e}\bra{e}
\end{align}
We call $\ket{g}$ the ground state and $\ket{e}$ the excited state. 


Classically the electric field $\vec{E}$ is given
\begin{align}
  \vec{E} = E_0 (\vec{\epsilon}\exp(i[\omega t - \vec{k}\cdot\vec{n}r]) + \vec{\epsilon}^*\exp(-i[\omega t - \vec{k}\cdot\vec{n}r])
\end{align}

We assume that we may treat the atom's center of mass classically and that the atom is positioned at $r=0$.

We need to write the dipole Hamiltonian, $H_d$, in terms of our system basis $\{\ket{g}, \ket{e}\}$
As the position operator $\vec{r}$ has odd parity, [??], the diagonal elements $\bra{g}H_d\ket{g}$ and $\bra{e}H_d\ket{e}$ must both be equal to $0$. As the Hamiltonian is Hermitian there is only one element left to specify:
\begin{align}
  \bra{e}H_d\ket{g} = \bra{g}H_d\ket{e}^* = E_0e\vec{r}_{eg}\cdot(\vec{\epsilon}\exp(i\omega t) + \vec{\epsilon}^*\exp(-i\omega t))
\end{align}
where $\vec{r}_{eg} = \bra{e}\vec{r}\ket{g}$

Over all this leaves us with the Hamiltonian
\begin{align}
  H =
  \begin{bmatrix}
    -\omega_0/2 & eE_0\vec{r}_{eg}^*\cdot(\vec{\epsilon}^*\exp(-i\omega t) + \vec{\epsilon}\exp(i\omega t)) \\
    eE_0\vec{r}_{eg}\cdot(\vec{\epsilon}\exp(i\omega t) + \vec{\epsilon}^*\exp(-i\omega t)) & +\omega_0/2
  \end{bmatrix}
\end{align}

Still time dependent. Make a unitary transformation to rotate with the EM field:
\begin{align}
  U(t) =
  \begin{bmatrix}
    e^{i\omega t/2} & 0 \\
    0 & e^{-i\omega t/2}
  \end{bmatrix}
\end{align}

This gives us
\begin{align}
  H =
  \begin{bmatrix}
    -(\omega_0-\omega)/2 & eE_0\vec{r}_{eg}^*\cdot(\vec{\epsilon}^* + \vec{\epsilon}\exp(2i\omega t)) \\
    eE_0\vec{r}_{eg}\cdot(\vec{\epsilon} + \vec{\epsilon}^*\exp(-2i\omega t)) & +(\omega_0-\omega)/2
  \end{bmatrix}
\end{align}

We now make the rotating wave approximation, neglecting the fast rotating terms $e^{2i\omega t}$ and $e^{-2i\omega t}$ on the basis that they will be far off-resonance and their contribution will average to zero over time. 

We are left with
\begin{align}
  H =
  \begin{bmatrix}
    -\omega_0/2 & eE_0\vec{r}_{eg}^*\cdot\vec{\epsilon}^* \\
    eE_0\vec{r}_{eg}\cdot\vec{\epsilon} & +\omega_0/2
  \end{bmatrix}
\end{align}
which is of the form of the Rabi Hamiltonian (\ref{rabi_hamiltonian}). An atom in a cavity interacting with an EM field will oscillate.

If we want to look at the emission and absorbtion of single photons we must instead use the fully quantised field. Quantum mechanically an electric field with a single mode and of definite polarisation is given
\begin{align}
  \vec{E} = E_0 (\vec{\epsilon}\an{a}+ \vec{\epsilon}^*\cre{a}).
\end{align}

With $\vec{r}_{eg}$ defined as before we write the dipole interaction term in the system basis:
\begin{align}
  H_d = E_0 e \vec{r}_{eg}\cdot(\vec{\epsilon}\an{a} + \vec{\epsilon}^*\cre{a})\sp + E_0 e \vec{r}_{eg}^* \cdot(\vec{\epsilon}\an{a} + \vec{\epsilon}^*),
\end{align}
where we have used spin language to describe atomic transitions, $\ket{e}\bra{g} \equiv \sp$ and $\ket{g}\bra{e} \equiv \sm$. We use this to rewrite the total Hamiltonian
\begin{align}
  H = \frac{\hbar \omega_0}{2} \sz + \hbar \omega \cre{a}\an{a} + g\an{a} \sp + g^{*}\cre{a} \sm + \gamma\an{a} \sm + \gamma^{*}\cre{a} \sp. 
\end{align}

If we have that $\omega \approx \omega_0$ the $\gamma$ terms will be highly off-resonant. Ignoring these terms is equivalent to making the RWA and gives
\begin{align}
  H = \frac{\hbar \omega_0}{2} \sz + \hbar \omega \cre{a}\an{a} + g\an{a} \sp + g^{*}\cre{a} \sm.
\end{align}
This is the well known \textit{Jaynes-Cummings Hamiltonian}. If we consider a basis of \textit{dressed states}, where we let $\ket{g, n} = \ket{g}\otimes\ket{n}$ and similarly for $\ket{e, n}$, we find that the Jaynes-Cummings Hamiltonian looks like the Rabi Hamiltonian on closed subspaces $\{\ket{g, n}, \ket{e, n-1}\}$ - the joint system oscillates between a state where it is relaxed and a state where it is excited but the mode contains one fewer excitation.

\subsection{Raman procedure}

We now consider a three level system $\{\ket{0}, \ket{1}, \ket{e}\}$ where the transitions $\ket{0} \leftrightarrow \ket{e}$ and $\ket{1} \leftrightarrow \ket{e}$ are optically active. Using the semi-classical approach, as at the beginning of the previous section, we can write the Hamiltonian
\begin{align}
  H=\hbar
  \begin{bmatrix}
    0 & 0 & \Omega_1 \cos\omega_1 t \\
    0 & \delta & \Omega_2 \cos\omega_2 t \\
    \Omega_1 \cos\omega_1 t & \Omega_2 \cos\omega_2 t & \omega
  \end{bmatrix},
\end{align}
where we have assumed for simplicity that for both optically active transitions we have that $\vec{\epsilon}\cdot \vec{r}_{eg} = \vec{\epsilon}^*\cdot \vec{r}_{eg} = \Omega$ for some real $\Omega$ , so that we can we can write the oscillitory behaviour in terms of a single cosine.

We next move to the rotating frame with the transformation
\begin{align}
  U(t) = 
  \begin{bmatrix}
    1 & 0 & 0 \\
    0 & e^{i(\omega_1 - \omega_2)t} & 0 \\
    0 & 0 & e^{\omega_1 t}
  \end{bmatrix},
\end{align}
so that the Hamiltonian becomes
\begin{align}
  \tilde{H}=\frac{\hbar}{2}
  \begin{bmatrix}
    0 & 0 & \Omega_1(1+e^{2i\omega_1 t}) \\
    0 & 2(\delta + \omega_2 - \omega_1) & \Omega_2 (1+e^{2i\omega_2 t}) \\
    \Omega_1(1+e^{2i\omega_1 t}) & \Omega_2 (1+e^{2i\omega_2 t}) & 2(\omega - \omega_1)
  \end{bmatrix}.
\end{align}

After we make the RWA to elimate the off-resonant $e^{\pm2\omega_2 t}$ terms, we are left with 
\begin{align}
  \tilde{H}=\frac{\hbar}{2}
  \begin{bmatrix}
    0 & 0 & \Omega_1 \\
    0 & 2(\nu_1 - \nu_2) & \Omega_2  \\
    \Omega_1 & \Omega_2  & 2\nu_1
  \end{bmatrix}.
\end{align}

We then restrict to the case where $\nu_1 = \nu_2 =: \Delta$, which corresponds to tuning the frequencies $\omega_1$ and $\omega_2$ so they are both detuned from resonance by the same amount. We spectrally decompose the Hamiltonian
\begin{align}
  H = \lambda_d \ket{d}\bra{d} + \lambda_{+}\ket{+}\bra{+} + \lambda_{-}\ket{-}\bra{-},
\end{align}
where 
\begin{align}
  \lambda_d &= 0 \\
  \lambda_\pm &= \Delta \pm \sqrt{\Delta^2 + \Omega_1^2 + \Omega_2^2}
\end{align}
and
\begin{align}
  \ket{d} &%= \frac{1}{n_d}\left(\Omega_2\ket{0} - \Omega_1\ket{1}\right)
  =\cos\theta \ket{0} - \sin\theta\ket{1} \\
  \ket{+} &%= \frac{1}{n_{+}}\left(\Omega_1\ket{0} + \Omega_2\ket{1} + \lambda_{+}\ket{e}\right)
  =\sin\phi \sin\theta \ket{0} + \sin\phi \cos\theta\ket{1}  + \cos\phi\ket{e}\\
  \ket{-} &%= \frac{1}{n_{-}}\left(\Omega_1\ket{0} + \Omega_2\ket{1} + \lambda_{-}\ket{e}\right)
  =\cos\phi \sin\theta \ket{0} + \cos\phi \cos\theta\ket{1}  - \sin\phi\ket{e}
\end{align}
with
\begin{align}
  \theta &= \arctan\left(\frac{\Omega_1}{\Omega_2}\right)\\
  \phi &= \arctan\left(\frac{\Omega_1^2 + \Omega_2^2}{\lambda_{+}}\right).
\end{align}

In the limit where $\Delta \gg \{\Omega_1, \Omega_2\}$ we get $\ket{+} \approx \ket{e}$ and $\ket{-} \approx \sin\theta\ket{0} + \cos\theta\ket{1}$, so if the system starts in the space spanned by $\{\ket{0}, \ket{1}\}$ it will stay there. In terms of this reduced basis the Hamiltonian becomes
\begin{align}
  H &\approx \Delta\left(1 - \sqrt{1 + \frac{\Omega_1^2 + \Omega_2^2}{\Delta^2}}\right)\ket{-}\bra{-} \\
   &\approx -\frac{1}{4\Delta}
  \begin{bmatrix}
    \Omega_1^2 & \Omega_1 \Omega_2 \\
    \Omega_1\Omega_2 & \Omega_2^2
  \end{bmatrix},
\end{align}
which has the same form as the Rabi Hamiltonian (\ref{rabi_hamiltonian}). Oscillation with angular frequency $\Delta = (\Omega_1^2 + \Omega_2^2)/4\nu$ between the states $\ket{0}$ and $\ket{1}$ is induced via the third state.


\section{Phonon-matter interactions}

When dealing with solid state systems an important source of decoherence comes from interaction with lattice vibrations. Following a similar pattern to the quantisation of the electromagnetic field, we can quantise the vibrational modes of a lattice, in terms of \textit{phonons}, quanta of vibrational energy, giving Hamiltonian
\begin{align}
  H_E = \sum_\alpha \omega_\alpha \cre{b}_\alpha \an{b}_\alpha,
\end{align}
where $\cre{b}$ and $\an{b}$ are creation and annihilation operators, satisfying the normal commutation relations. We usually assume that the phonon bath exists in a thermal state
\begin{align}
  \rho_E = \frac{1}{Z_\beta} e^{-\beta H_E}
\end{align}
where $\beta = 1/(kT)$ ans $Z_\beta$ is a normalisation factor (the parition function from statistical mechanics).

Phonons couple to any states of our system that lead to a distortion in the atomic lattice. We restrict ourselves to looking at relatively low-energy states, coupling only to long-wavelength, accoustic phonons. There are two dominant coupling mechanisms in this case: \textit{deformation potential} coupling, $D$, and \textit{piezoelectric coupling}, $P$, \cite{mahan} leading to the interaction Hamiltonian
\begin{align}
  H = \sum_\vec{q} \sqrt{\frac{\hbar}{2\mu V\omega_\vec{q}}} \left(D \vert \vec{q}\vert + iP \right)\hat{g}(\vec{q})\left(\an{a}_\vec{q} + \cre{a}_{-\vec{q}} \right).
\end{align}

As a concrete example we look at the Raman process from the previous section. We envisage the lambda system embedded in some solid state system and assume that phonons couple only to the excited state $\ket{e}$ so that the interaction Hamiltonian becomes
\begin{align}
  H = \ket{e}\bra{e}\sum_\vec{q} g_\vec{q}\left(\an{a}_\vec{q} + \cre{a}_{-\vec{q}}\right).
\end{align}
This Hamiltonian is appropriate for certain quatum dot systems \cite{erik_qd_paper} and uses the fact that in these systems the deformation potential dominates \cite{ep76, ep138}.

Following the technique given in chapter \ref{ch:QuantumDynamics} we move to the interaction picture and decompose the system part of the interaction Hamiltonian, $\ket{e}\bra{e}$ into eigenoperators of the system Hamiltonian:
\begin{align}
  \tilde{H} = \left(P_0 + P_\Lambda e^{-i\Delta t}e^{\Lambda t} + P^\dagger_\Lambda e^{-i\Delta t} e^{i\Lambda t}\right)\sum_\vec{q}g_\vec{q}\left(\an{a}_\vec{q}e^{-i\omega_\vec{q} t} + \cre{a}_\vec{q} e^{i\omega_\vec{q} t} \right)
\end{align}
where $P_0$ and $P_\Lambda$ are given in terms of the eigenvectors from the previous section
\begin{align}
  P_0 &= \cos^2\phi \ket{+}\bra{+} + \sin^2\phi \ket{-}\bra{-} \\
  p_\Lambda &= -\sin\phi \cos\phi \ket{-}\bra{+}.
\end{align}
By comparing with eq. (\ref{xxx}) we identify
\begin{align}
  B_0(t) &= \sum_\vec{q} g_\vec{q} \left(\an{a}_\vec{q} e^{-i\omega_\vec{q} t} + \cre{a}_\vec{q} e^{i\omega_\vec{q} t}\right)
  B_\Lambda(t) &= e^{-i\Delta t}\sum_\vec{q} g_\vec{q} \left(\an{a}_\vec{q} e^{-i\omega_\vec{q} t} + \cre{a}_\vec{q} e^{i\omega_\vec{q} t}\right).
\end{align}

\begin{align}
  \Gamma_{\alpha\beta}(\omega) = \frac{1}{\hbar^2}\int_0^\infty ds e^{i\omega s}\langle B^\dagger_\beta(t) B_\alpha(t-s)\rangle
\end{align}

In the thermal state we can calculate the correlation functions explicitly
\begin{align}
  \langle \an{b}_i \an{b}_j \rangle &= 0 \\
  \langle \cre{b}_i \cre{b}_j \rangle &= 0 \\
  \langle \an{b}_i \cre{b}_j \rangle &= \delta_{ij} \left( N(\omega_i) +1 \right) \\
  \langle \cre{b}_i \an{b}_j \rangle &= \delta_{ij} N(\omega_i)
\end{align}
where
\begin{align}
  N(\omega) = \frac{1}{e^{\beta \omega} - 1}
\end{align}
is the occupation of state $i$. This allows us to 
\begin{align}
  \gamma_\alpha(\omega) = \left\{
    \begin{array}{l l}
    J(\omega)\left(N(\omega) + 1\right) & \qquad \text{if $\omega > 0$}\\
    J(\omega)N(\omega) & \qquad \text{if $\omega \leq 0$}
  \end{array} \right.
\end{align}
where the \textit{spectral density} is given by
\begin{align}
  J(\omega) = 2\pi \sum_\vec{q} |g_\vec{q}|^2 \delta(\omega-\omega_\vec{q}).
\end{align}
The spectral density gives an indication of how much coupling affinity our system has for the phonons of frequency $\omega$.

We end up with the phonon dissipator term
\begin{align}
  D_\text{ph}(\rho) = J(\Lambda)\left[\left(N(\Lambda) + 1)D[P_\Lambda]\rho\right) + N(\Lambda)D[P^\dagger_\Lambda]\rho \right],
\end{align}
where $D[L]\rho = L\rho L^\dagger -1/2(L^\dagger L\rho + \rho L^\dagger L)$.

