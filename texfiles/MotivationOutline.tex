\chapter{Motivation and outline} 
\label{ch:Motivation}

\section{Motivation}

The theory of quantum mechanics is arguably the biggest scientific breakthrough of the 20th Century. The seemingly simple postulates that model behaviour of atomic-scale systems give rise to deep-reaching and counter-intuitive consequences .
The philosophical implications of quantum mechanics in terms of the apparent lack of determinism in the universe we inhabit, the various interpretations of the quantum superposition and wavefunction collapse, and the apparent paradoxes afforded by phenomenum of quantum entanglement have, and continue, to occupy physicists and philosopers alike.

It is somewhat remarkable that a theory provoking such controversy has been such a success experimentally and technologically. When the philosophical implications are put aside and the laws accepted, we are left with an incredibly powerful tool for modelling and reasoning about the behaviour of atomic-scale systems. Many of the key technological advances responsible for shaping our world originated in insights aforded by quantum mechanics: the transistor depends on the quantm mechanical treatment of solid state physics that explains the PN junction; the laser depends both on solid state physics and the quantum mechanical description of light. Without quantum mechanics the word we see around us would be unrecogniseable.

Like quantum mechanics itself, \textit{quantum information processing} touches on issues of fundamental physics and philosophy, whilst also concerning itself with practical technological advancement. In many ways, quantum mechanics is best understood in the language of information theory. The well-known form of Heisenberg's Uncertainty Principle - that by measuring the state of a system we change it - is already in this language, relating our change in knowledge about the system to a physical change in its state. The duality between information gain and quantum wave function collapse is strong enough to be of use in forming a working intuition of quantum systems and informally reasoning about their behaviour.


Hmm.. 

\section{Outline}

We start this thesis with a general introduction 

Chapter \ref{ch:SpinAmplification} contains a 

In Chapter \ref{ch:SpinAmplification}, we introduce 

We proceed by  in Chapter \ref{ch:SpinAmplification}. 

In Chapter \ref{ch:SpinAmplification}, we provide 

Chapter \ref{ch:SpinAmplification} is concerned with 

Chapter \ref{ch:SpinAmplification} presents 

Finally, in Chapter \ref{ch:SpinAmplification} 

As the research chapters address a rather broad range of different questions in different physical systems, they will each have their own conclusion section in which we summarise key results of the study and discuss interesting remaining questions and avenues for further work.


