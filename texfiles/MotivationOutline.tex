\chapter{Introduction} 
\label{ch:Motivation}

The theory of quantum mechanics is arguably the biggest scientific breakthrough of the 20th Century. The seemingly simple postulates that model behaviour of atomic-scale systems give rise to deep-reaching and counter-intuitive consequences. The philosophical implications of quantum mechanics in terms of the apparent lack of determinism in the universe we inhabit, the various interpretations of the quantum superposition and wavefunction collapse, and the apparent paradoxes afforded by phenomenon of quantum entanglement have occupied, and continue to occupy, physicists and philosophers alike.

It is remarkable that a theory provoking such controversy has been such a success experimentally and technologically. When the philosophical implications are put aside and the laws accepted, we are left with an incredibly powerful tool for modelling and reasoning about the behaviour of atomic-scale systems. Many of the key technological advances responsible for shaping our world originated in insights afforded by quantum mechanics: the transistor depends on the quantum mechanical treatment of solid state physics that explains the PN junction; the laser depends both on solid state physics and the quantum mechanical description of light. Without the theory of quantum mechanics, the world we see around us would be unrecognisable.

Quantum Information Processing has the potential to introduce a new wave of technological advances. The theory of the field is well developed and provides  powerful new applications of quantum mechanics. The challenge of realising these applications is now largely the challenge of building robust devices that operate on the nanoscale, with an ability to protect information stored in the quantum wavefunction. In this thesis we examine robust techniques for spin measurement, photon preparation and error correction.

We start with a general introduction to Quantum Information Processing in Chapter \ref{ch:QuantumInformationProcessing}, surveying different areas of research and detailing important recent developments. The next two chapters introduce the theoretical framework and techniques used later in the thesis. Chapter \ref{ch:QuantumDynamics} covers relevant parts of the theory of Open Quantum Systems and Chapter \ref{ch:Light} contains topics we will use when modelling interactions between matter and light.

In Chapter \ref{ch:SpinAmplification} we present a robust spin-measurement procedure, using an amplification approach: the state of the spin is propagated over a two-dimensional array to a point where it can be measured using standard macroscopic state measurement techniques. The work is an example of how simple propagation rules can lead to predictable macroscopic behaviour and the techniques should be applicable in other state propagation schemes.

Chapter \ref{ch:SinglePhotonSource} is concerned with the creation of a reliable single photon source in the face of phonon decoherence. We introduce a master equation technique for quantum jump situations that should have wide application in other situations.

Finally, in Chapter \ref{ch:SurfaceCodes} we look at toric error correcting codes. The chapter is self contained, starting with an overview and review of the basic toric code and current decoding approaches. We then go on to investigate the power of small scale toric codes and determine the minimum size of code necessary for a first experimental demonstration of toric coding power.

As the research chapters address a rather broad range of different questions in different physical systems, they will each have their own conclusion section in which we summarise key results of the study and discuss interesting remaining questions and avenues for further work.


