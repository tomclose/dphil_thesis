\chapter{Introduction to quantum information processing} 
\label{ch:Introduction}

\section{Quantum Information Processing}

Quantum Information Processing (QIP) is the study of how quantum mechanical systems can be used to perform information processing tasks. 

\subsection{Quantum Computing}

expoit the complexity of a many particle wave function to solve a computational problem

\cite{feynman_82} exponentially growing number of paths in a quantum system makes it difficult to simulate classically
\cite{deutsch_85} universal quantum computer capable of simulating any system more powerful than classical
\cite{deutsch_jozsa_92} deutsch josza first quantum algoritm

\cite{shors_algorithm}

information leaking from system - destroys wavefunction - decoherence

error correction \cite{knill_correction} errors per gate as high as 3\% with concatenated codes
 
\cite{nature_cq_review_10}

\subsection{Quantum Communication}

\subsection{Quantum Simulation}
\cite{simulation_ion_review} Shors algorithm for large numbers around 10s of 1000s qubits
but even small systems hard to simulate classically
\cite{feynman_82} proposal to use quantum systems for simulations, quantum systems exibit an exponentially growing number of paths

simulators more tolerant of mistakes

\cite{lloyd_universal_simulators} existance of universal quantum simulators for systems with local interactions, 1996

systems like photosynthesis 

\section{Models of Quantum Computing}

\subsection{Circuit Model}

\subsection{Adiabatic Quantum Computing}
\cite{adiabatic_equivalent} 2007

\subsection{Measurement Based Quantum Computing}
\cite{one_way_qc} Raussendorf Briegel 2001

\subsection{Topological Quantum Computing}


\section{Distributed Computing Architectures}

\subsection{Remote entanglement generation}


\section{Candidate Quantum Systems}

Any system that exhibits quantum mechanical behaviour has the potential to be used for QIP. There are a plethora of different approaches currently under investigation. Here we briefly survey the most well known of the approaches and attempt to detail any notable progress that has been made.

\textit{Nuclear magnetic resonance}
existing well-understood technology
\cite{nmr_proposal_chuang_97}
lacks entanglement \cite{nmr_pseudo_pure}
NMR not-scalable, favourable for small-scale quantum algorithms
\cite{nmr_143_factorization} Factorise 143 using adiabatic algorithm, 2011

\textit{Ion traps}
\cite{cirac_zoller_ion_trap_proposal_95}
\cite{first_ion_trap_wineland_98}
scaling difficulties for cooling large numbers of atoms
move between different zones so gates can work with small numbers of atoms
\cite{ion_trap_14_qubits} 14 quibit entanglement, 2011, Blatt
\cite{ion_trap_simulator} simulation of open quantum system
\cite{ion_trap_digital_simulator}
optical lattice to trap 100s of atoms
\cite{ion_trap_magnetism_simulator} 2d system of 100s of ions simulating ising interaction - works well for classical simulatable weak fields, potential for strong

2005 University of Michigan produced a chip based ion trap \cite{?}

\textit{Cavity QED}
?

\textit{Quantum dots}


\textit{Optical lattices}
Free from decoherence, polarisation rotations (1 qubit gates) simple
interactions a problem - non-linear effects weak
\cite{klm} - single photon sources, detectors, linear circuits, non-deterministic interactions
\cite{science_loqc_review} - theoretical advances to reduce resources
\cite{shor_chip_bristol} 4 optical qubits, factorise 15, 2009
\cite{shor_chip_bristol_2} 2 photons, to factorise 21, using qubit recycling, 2012
efficient detectors \cite{single_photon_detector_review_09} and sources \cite{single_photon_source_review_04}
photon loss a problem

\textit{NV Centres}
http://www.nature.com/nphys/journal/v9/n3/full/nphys2563.html#ref2
protected from magnetic noise, active colour centers
\cite{two_qubit_nv}
\cite{nv_entanglement_hanson} entanglement of NV with C13 - non-deterministic
\cite{nv_entanglement_dolde} deterministic entanglement between neighbouring pairs of NVs
\cite{remote_nv_entanglement_hanson} non-deterministic entanglement between NVs on different pieces of diamond

\textit{Superconducting qubits}
\cite{two_qubit_chip_yale} 2 qubits, Deutsch-Josza, Grover, 2009
\cite{dwave_annealing} quantum annealing - finding ground state of a system, 2011


\cite{silicon_qubit} 2012 a qubit in silicon, 200us spin coherence time electron?, hours for nuclear spin at room temp?, atom + electron, charge to spin conversion
\cite{silicon_seconds}


----------------------------

At the heart of quantum computing lies the observation that it is very hard to simulate a quantum mechanical system on a classical computer. Due to the principle of quantum superposition - that a quantum system can exist in many states at once - a classical computer is forced to compute many possible trajectories of a system in order to make an accurate statement about the actual motion. As a system grows in size, tracking the superpostion states requires an exponential amount of work, making exact simulation unfeasable. It is thought that it was Feynman in the early 1980s \cite{feynman82} who first suggested turning this idea on its head: that a computer that followed the laws of quantum mechanics might have processing capabilities that would fundamentally outstrip those of its classical counterparts. 

\section{Measurement Based Quantum Computing}

Quantum computing is usually presented in terms of the circuit model. 

Measurement based quantum computation (MBQC) or one-way quantum computation is a computing paradigm proposed by Raussendorf and Briegel \cite{raussendorf01}. A computation is performed by making irreversible measurements on a highly entangled quantum state.

The approach separates the computation into two separate stages: the creation of a suitable entangled state; and the implementation of the algorithm by making local measurements that consume this state. This is both practically and conceptually useful. Practically it separates the creation of entanglement from the running of the algorithm and thus allows new possibilities in the entanglement mechanism, such as the use of mechanisms with a high failure probablilties. Conceptually it allows us to view entanglement as a resource to be consumed throughout the computation and to ask questions about how to quantify entanglement and relate this to the computations we can perform. 

The key concept behind MBQC is that a unitary gate on a quantum state can be implemented as a specially chosen measurement on a larger entangled state. For example, consider the state $\bra{\psi}=\alpha\bra 0+\beta\bra 1$ entangled with an ancilla qubit using a CZ gate. We will measure this state in the $\bra 0\pm e^{i\phi}\bra 1$ basis. Rewriting the state in this basis:
\begin{equation}
\alpha\bra{00}+\alpha\bra{01}+\beta\bra{10}-\beta\bra{11}=(\bra 0+e^{i\phi}\bra 1)(\alpha\bra ++\beta e^{i\phi}\bra -)+(\bra 0-e^{i\phi}\bra 1)(\alpha\bra +-\beta e^{i\phi}\bra -)
\end{equation}
so, we have moved the state $\bra{\psi}$ onto the ancilla and performed the rotation $HZ^{m}R_{z}(\phi)$, where $m$ is the measurement outcome, i.e. $0$ if the $\bra 0+e^{i\phi}\bra 1$ state was measured and $1$ if the $\bra 0-e^{i\phi}\bra 1$ state was measured.

Although we have only performed a rotation around the $Z$ axis here, by concatenating such operations we can perform arbitary single qubit rotations. By starting with a network of entangled qubits we can not only perform an arbitrary number of these single qubit rotations but also effectively implement two qubit gates by exploiting the entanglement already existing in the network.

In this manner we can see that the MBQC model is able to implement any quantum algorithm possible in the circuit model and thus is universal for quantum computing. Throughout the calculation it is necessary to keep track of the extra $Z$ rotations that are dependent on the measurement outcomes and modify future measurement bases accordingly. This imposes a time ordering on the measurements which prevents all the measurements being carried out independently and the resulting paradox that all algorithms could be performed in $O(1)$ time.  


