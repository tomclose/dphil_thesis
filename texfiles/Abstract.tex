\vspace{-2cm}
\enlargethispage{2cm}
\begin{center}

  {\Huge Robust Quantum Phenomena \\[0.0cm] 
               for \\[0.3cm] 
               Quantum Information Processing}\\[1cm]

  {\large Tom Close\\
               Oriel College, University of Oxford\\
               Hilary Term 2013\\[1cm]
  }

\end{center}
 
\section*{Abstract of thesis submitted for the degree of \mbox{Doctor of Philosophy}} 

This thesis is concerned with investigating and proposing robust components for quantum information processing systems. We examine three different areas covering techniques for spin measurement, photon preparation and error correction.

The first research chapter presents a robust spin-measurement procedure, using an amplification approach: the state of the spin is propagated over a two-dimensional array to a point where it can be measured using standard macroscopic state measurement techniques. Even in the presence of decoherence, our two-dimensional scheme allows a linear growth in the total spin polarisation - an important increase over the $\sqrt{t}$ obtainable in one-dimension. The work is an example of how simple propagation rules can lead to predictable macroscopic behaviour and the techniques should be applicable in other state propagation schemes. 

The next chapter is concerned with strategies for obtaining a robust and reliable single photon source. Using a microscopic model of electron-phonon interactions and a quantum master equation, we examine phonon-induced decoherence and assess its impact on the rate of production, and indistinguishability, of single photons emitted from an optically driven quantum dot system. We find that, above a certain threshold of desired indistinguishability, it is possible to mitigate the deleterious effects of phonons by exploiting a three-level Raman process for photon production. We introduce a master equation technique for quantum jump situations that should have wide application in other situations.

The final chapter focusses on toric error correcting codes. Toric codes form part of the class of surface codes that have attracted a lot of attention due to their ability to tolerate a high level of errors, using only local operations. We investigate the power of small scale toric codes and determine the minimum size of code necessary for a first experimental demonstration of toric coding power.


